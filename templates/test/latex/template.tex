\documentclass[journal]{IEEEtran}

\usepackage{cite}
\usepackage{amsmath,amssymb,amsfonts}
\usepackage{algorithmic}
\usepackage{graphicx}
\usepackage{textcomp}
\usepackage{xcolor}
\usepackage{hyperref}

\def\BibTeX{{\rm B\kern-.05em{\sc i\kern-.025em b}\kern-.08em\TeX}}

\begin{document}

\title{Your Paper Title}

\author{Author 1,~\IEEEmembership{Member,~IEEE,}
        Author 2,~\IEEEmembership{Fellow,~OSA,}
        and~Author 3,~\IEEEmembership{Life~Fellow,~IEEE}% <-this % stops a space
\thanks{Author 1 is with the Department of XYZ, University ABC, City, Country (e-mail: author1@xyz.edu).}%
\thanks{Author 2 is with the Department of PQR, University DEF, City, Country (e-mail: author2@pqr.edu).}%
\thanks{Author 3 is with the Department of LMN, University GHI, City, Country (e-mail: author3@lmn.edu).}%
}

\markboth{Journal of IEEE Transactions on XYZ,~Vol.~X, No.~X, Month~Year}%
{Author 1 \MakeLowercase{\textit{et al.}}: Your Paper Title}

\maketitle

\begin{abstract}
This document provides a basic template for writing papers in the IEEE Transactions format. The abstract should be a brief summary of your paper, usually not exceeding 200 words.
\end{abstract}

\begin{IEEEkeywords}
IEEE, template, transactions, paper, LaTeX
\end{IEEEkeywords}

\section{Introduction}
The introduction should clearly state the problem that you are addressing and give an overview of the contributions made in the paper.

\section{Related Work}
Provide an overview of relevant prior work in this section.

\section{Methodology}
Describe your methods, models, algorithms, or techniques in this section.

\section{Results}
Discuss the results obtained, including tables, graphs, and other representations.

\section{Conclusion}
Summarize the main findings and potential directions for future research.

\appendices
\section{Appendix A}
Appendixes, if any, are optional.

\section*{Acknowledgment}
The authors would like to thank...

\begin{thebibliography}{00}
\bibitem{b1} G. O. Young, ``Synthetic structure of industrial plastics,'' in \emph{Plastics}, 2nd ed., vol. 3. New York, NY, USA: McGraw-Hill, 1964, pp. 15--64.
\bibitem{b2} W.-K. Chen, \emph{Linear Networks and Systems}, 1st ed. Belmont, CA, USA: Wadsworth, 1993, pp. 123--135.
\end{thebibliography}

\end{document}
