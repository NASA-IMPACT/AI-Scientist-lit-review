\documentclass[journal]{IEEEtran}

\usepackage{cite}
\usepackage{amsmath,amssymb,amsfonts}
\usepackage{algorithmic}
\usepackage{graphicx}
\usepackage{textcomp}
\usepackage{xcolor}
\usepackage{hyperref}

\def\BibTeX{{\rm B\kern-.05em{\sc i\kern-.025em b}\kern-.08em\TeX}}

\begin{document}

\title{Automated Literature Review on [Topic]: A Survey of Techniques and Models}

\author{Author 1,~\IEEEmembership{Member,~IEEE,}
        Author 2,~\IEEEmembership{Member,~IEEE,}
        and~Author 3,~\IEEEmembership{Fellow,~IEEE}% <-this % stops a space
\thanks{Author 1 is with the Department of [Department Name], [University Name], City, Country (e-mail: author1@abc.edu).}%
\thanks{Author 2 is with the Department of [Department Name], [University Name], City, Country (e-mail: author2@abc.edu).}%
\thanks{Author 3 is with [Organization], City, Country (e-mail: author3@abc.edu).}%
}

\markboth{AUTOMATED LIT REVIEW}%
{Author 1 \MakeLowercase{\textit{et al.}}: Automated Literature Review on [Topic]}

\maketitle

\begin{abstract}
This paper presents an automated literature review on [topic], highlighting key advancements, models, and approaches used in [specific field]. By focusing on [key areas such as techniques, benchmarks, etc.], this review provides insights into the performance and applications of various models. The review synthesizes methodologies, evaluations, and future directions from recent works, identifying gaps in the current research landscape.
\end{abstract}

\begin{IEEEkeywords}
Automated Literature Review, [Add relevant keywords], Models, Techniques, Evaluation.
\end{IEEEkeywords}

\section{Introduction}
In recent years, [field/topic] has seen significant advancements, particularly with the rise of models such as [mention any models if needed]. However, despite these improvements, there are still challenges in [specific aspect], such as [mention any problem]. This paper provides an automated review of the current state of the field, summarizing key contributions and identifying areas for future research.

\section{Background and Motivation}
The development of [model/technique] has been instrumental in [specific task]. Models like [model name] have been utilized in [specific application]. Despite this, further advancements are necessary to address [mention challenge]. This section covers the foundational work that paved the way for these innovations.

\section{Methodology for Literature Review}
This automated review draws from various sources such as [specific databases]. The papers are categorized based on their relevance to [specific topics]. This section outlines the methodology for paper selection, categorization, and key metrics used in the evaluation.

\section{Key Models and Techniques}
\subsection{Model A}
[Describe Model A briefly, its origin, and key contributions.]

\subsection{Model B}
[Describe Model B, its relevance to the topic, and areas of performance.]

\section{Evaluation and Benchmarking}
The performance of the models discussed is evaluated across various benchmarks, such as [mention benchmarks]. Table \ref{tab:performance} summarizes the performance metrics for key models.

\begin{table}[h!]
    \centering
    \caption{Performance Comparison of Models on Benchmark Tasks}
    \begin{tabular}{|c|c|c|c|}
    \hline
    Model & Dataset & Metric & Score \\
    \hline
    Model A & Dataset 1 & F1 Score & 0.92 \\
    Model B & Dataset 2 & Accuracy & 0.89 \\
    \hline
    \end{tabular}
    \label{tab:performance}
\end{table}

\section{Unique Contributions and Strengths}
The key strengths of the reviewed models include [specific advantages such as efficiency, accuracy, domain specificity, etc.]. For example, [model A] excels at [mention task] due to [specific feature].

\section{Gaps and Future Directions}
Although the reviewed models show significant progress, there are still gaps in [mention any shortcomings]. Future research could focus on [mention future directions such as optimization, new datasets, etc.].

\section{Conclusion}
This review provides an overview of the advancements in [field], particularly focusing on models like [Model A] and [Model B]. Despite their strengths, further research is required to bridge the gaps and improve [mention key areas].

\section*{Acknowledgment}
The authors would like to thank [mention any organizations or individuals] for their contributions.

\begin{thebibliography}{00}
\bibitem{ref1} Author 1, Author 2, ``Title of the paper,'' \textit{Journal Name}, vol. x, no. x, pp. xxx-xxx, Year.
\bibitem{ref2} Author 3, Author 4, ``Title of another paper,'' \textit{Journal Name}, vol. x, no. x, pp. xxx-xxx, Year.
\end{thebibliography}

\end{document}
